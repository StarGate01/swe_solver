\documentclass[shortpres]{beamer}
\usetheme{CambridgeUS}

% Depending on build configuration, one of these packages will
% enable unicode
%\usepackage[utf8]{inputenc}
\usepackage{fontspec}


%Images
\usepackage{graphicx, svg}
\usepackage{caption}

%Showing movies and gifs
%\usepackage{animate, media9, movie15}
\usepackage{animate}
\usepackage{array}
\usepackage{subfig}
\usepackage{multicol}
\usepackage{color}
\usepackage{pgfplots}
\usepackage{xmpmulti}

\usepackage{algorithm,algpseudocode}  %for algorithm environment

\setbeamertemplate{footline}
{
b	\leavevmode%
	\hbox{%
		\begin{beamercolorbox}[wd=.333333\paperwidth,ht=2.25ex,dp=1ex,center]{author in head/foot}%
			\usebeamerfont{author in head/foot}\insertshortauthor%~~\beamer@ifempty{\insertshortinstitute}{}{(\insertshortinstitute)}
		\end{beamercolorbox}%
		\begin{beamercolorbox}[wd=.333333\paperwidth,ht=2.25ex,dp=1ex,center]{title in head/foot}%
			\usebeamerfont{title in head/foot}\insertshorttitle
		\end{beamercolorbox}%
		\begin{beamercolorbox}[wd=.333333\paperwidth,ht=2.25ex,dp=1ex,right]{date in head/foot}%
			\usebeamerfont{date in head/foot}\insertshortdate{}\hspace*{2em}
			\insertframenumber{} / \inserttotalframenumber\hspace*{2ex}
	\end{beamercolorbox}}%
	\vskip0pt%
}\part{title}
\beamertemplatenavigationsymbolsempty


%color specification---------------------------------------------------------------
\definecolor{TUMblue}{rgb}{0.00, 0.40, 0.74}
\definecolor{TUMgray}{rgb}{0.85, 0.85, 0.86}
\definecolor{TUMpantone285C}{rgb}{0.00, 0.45, 0.81}
\definecolor{lightblue}{rgb}{0.7529,0.8118,0.9333}

\setbeamercolor{block title}{fg=white, bg=TUMpantone285C}
\setbeamercolor{block body}{bg=lightblue}
\setbeamertemplate{blocks}[rounded][shadow=true]
%----------------------------------------------------------------------------------

\setbeamercolor{frametitle}{fg=TUMblue, bg=white}
\setbeamercolor{palette primary}{fg=TUMblue,bg=TUMgray}
\setbeamercolor{palette secondary}{use=palette primary,fg=TUMblue,bg=white}
\setbeamercolor{palette tertiary}{use=palette primary,fg=white, bg=TUMblue}
\setbeamercolor{palette quaternary}{use=palette primary,fg=white,bg=TUMpantone285C}


\setbeamercolor{title}{bg=white,fg=TUMblue}
\setbeamercolor{item projected}{use=item,fg=black,bg = lightblue}
\setbeamercolor{block title}{fg=black, bg=lightblue}
\setbeamercolor{block body}{bg=white}
\setbeamertemplate{blocks}[rounded][shadow=true]
%----------------------------------------------------------------------------------



%############### Self defined commands ##############################
\newcommand{\imgvoffset}{-20pt}
\newcommand{\texthoffset}{20pt}
\newcommand{\imgfullscale}{0.75}

\captionsetup[subfigure]{labelformat=empty}		%Disable enumeration of subfigures
%####################################################################

\usepackage{anyfontsize}

\title[{Tsunami simulation}]{Assignment 1}

\author[Bellamy, Honal, Wieser]{George Bellamy, Christoph Honal, Felix Wieser\\\vspace{10pt}{\small Bachelorpraktikum}}

\institute[TU M\"unchen]{Technical University of Munich}

\date{7. November 2017}

\begin{document}
\maketitle

\begin{frame}{Solver implementation: Core functionality}
	\begin{figure}
		\only<1>
		{
			\subfloat[FWave]{\includegraphics[clip, width=0.4\linewidth]{img/FWave.png}}
			\hspace{20pt}
			\subfloat[FCore]{\includegraphics[clip, width=0.4\linewidth]{img/FCore.png}}	
		}
		\only<2>
		{
			\subfloat[FCalc]{\includegraphics[clip, width=0.4\linewidth]{img/FCalc.png}}
			\hspace{20pt}
			\subfloat[vector2 of FStruct]{\includegraphics[clip, width=0.25\linewidth]{img/vector2.png}}
		}
	\end{figure}
	
	The core functionality is provided by the following classes
	\begin{tabular}{ll}
		FWave & Solver of the 2D-Problem\\
		FCore & Implementation of NetUpdates and eigenvalues computation\\
		\pause
		FCalc & Supplemental methods (average height, ...)\\
		FStruct & Vector2-Implementation and other structs
	\end{tabular}
\end{frame}

\begin{frame}{Solver implementation: Testing}
	\begin{multicols}{2}
		
		\begin{figure}
			\subfloat{\includegraphics[clip,width=0.7\linewidth]{img/FCoreTestSuite.png}}\\
			\subfloat{\includegraphics[clip,width=0.7\linewidth]{img/CSVParser.png}}
		\end{figure}
		
		\columnbreak
		
		\vspace*{30pt}
		Tests cover
		\begin{itemize}
			\item Core functionality
			\item Eigenvalue computation
			\item Supplemental methods in FCalc
			\item Comparison with precomputed data
		\end{itemize}
		\vfill
	\end{multicols}
\end{frame}

\begin{frame}{Shock-Shock Problem}
	\begin{figure}[t]
		\vspace{\imgvoffset}
		\includegraphics[clip, width=\imgfullscale\linewidth]{img/Shock-Shock.png}
		\caption*{Shock-Shock problem}
	\end{figure}
	A wave is created and moves outward from the middle point
\end{frame}


\begin{frame}{Shock-Shock: impact of water height}
	\begin{figure}[t]
		\vspace{\imgvoffset}
		\includegraphics[width=\imgfullscale\linewidth]{img/Shock_h10_t18.png}
		\caption*{}
	\end{figure}
	
	\begin{tabular}{m{3cm} m{\linewidth-5cm}}
		$
		\begin{matrix}
		h & = & 10\\
		hu_r/_l & = & 10
		\end{matrix}
		$
		&
		
		The \Delta waveheight is 1m \newline
		The wave is 40m wide after 18s
	\end{tabular}
\end{frame}

\begin{frame}{Shock-Shock: impact of water height}
	\begin{figure}[t]
		\vspace{\imgvoffset}
		\includegraphics[width=\imgfullscale\linewidth]{img/Shock_h50_t8-7.png}
		\caption*{}
	\end{figure}
	
	\begin{tabular}{m{3cm} m{\linewidth-5cm}}
		$
		\begin{matrix}
		h & = & 50\\
		hu_r/_l & = & 10
		\end{matrix}
		$
		&
		
		The \Delta waveheight is 0,5m \newline
		The wave is 40m wide after 8,7s
	\end{tabular}
\end{frame}



\begin{frame}{Shock-Shock: impact of different water speeds}
	\begin{figure}[t]
		\vspace{\imgvoffset}
		\includegraphics[width=\imgfullscale\linewidth]{img/Shock_hul50_t13.png}
		\caption*{}
	\end{figure}
	
	\begin{tabular}{m{3cm} m{\linewidth-5cm}}
		$
		\begin{matrix}
		h & = & 10\\
		hu_l & = & 50\\
		hu_r & = & 10
		\end{matrix}
		$
		&
		
		After 13s the wave has moved 5m to the right 
	\end{tabular}
\end{frame}


\begin{frame}{Rare-Rare Problem}
	\begin{figure}[t]
		\vspace{\imgvoffset}
		\includegraphics[clip, width=\imgfullscale\linewidth]{img/Rare-Rare.png}
		\caption*{Rare-Rare problem}
	\end{figure}
	%TODO: Add text
	The same observations through height- and speedchanges as in Shock-Shock can be observed here only inversed
\end{frame}

\begin{frame}{Dam-Break: impact of water height}
	\begin{figure}[t]
		\vspace{\imgvoffset}
		\includegraphics[width=\imgfullscale\linewidth]{img/Dam_hl8_hr5.png}
		\caption*{}
	\end{figure}
	
	\begin{tabular}{m{3cm} m{\linewidth-5cm}}
		$
		\begin{matrix}
		h_l & = & 8\\
		h_r & = & 5
		\end{matrix}
		$
		&
		
		Wave gets to 72m in 20s
	\end{tabular}
\end{frame}

\begin{frame}{Dam-Break: impact of water height}
	\begin{figure}[t]
		\vspace{\imgvoffset}
		\includegraphics[width=\imgfullscale\linewidth]{img/Dam_hl25_hr5.png}
		\caption*{ }
	\end{figure}
	
	\begin{tabular}{m{3cm} m{\linewidth-5cm}}
		$
		\begin{matrix}
		h_l & = & 25\\
		h_r & = & 5
		\end{matrix}
		$
		&
		
		Wave gets to 68m in 10s
		\newline \rightarrow higher left water level results in a faster wave
	\end{tabular}
\end{frame}


\begin{frame}{Dam-Break: Scenario 3}
	\begin{figure}[t]
		\vspace{\imgvoffset}
		\includegraphics[width=\imgfullscale\linewidth]{img/Dam_river.png}
		\caption*{}
	\end{figure}
	
	\begin{tabular}{m{3cm} m{\linewidth-5cm}}
		$
		\begin{matrix}
		h_l & = & 8\\
		h_r & = & 5\\
		hu_r & = & 10
		\end{matrix}
		$
		&
		
		A higher river speed reduces the wave height
	\end{tabular}
\end{frame}

\begin{frame}{Dam-Break: Scenario 4}
	\begin{figure}[t]
		\vspace{\imgvoffset}
		\includegraphics[width=\imgfullscale\linewidth]{img/Dam_highspeedriver.png}
		\caption*{}
	\end{figure}
	
	\begin{tabular}{m{3cm} m{\linewidth-5cm}}
		$
		\begin{matrix}
		h_l & = & 8\\
		h_r & = & 5\\
		hu_r & = & 30
		\end{matrix}
		$
		&
		
		Very high river speeds make the wave lower than the rivers initial height
	\end{tabular}
\end{frame}


\begin{frame}{Dam-Break: When does the water arrive?}
	\begin{figure}[t]
		\vspace{\imgvoffset}
		\includegraphics[width=\imgfullscale\linewidth]{img/4_dorf_arrival_as_graph.png}
		\caption*{Arrival of the wave at the village}
	\end{figure}
	The wave arrives after $\approx$ $7,3$ minutes at the village
\end{frame}

\end{document}