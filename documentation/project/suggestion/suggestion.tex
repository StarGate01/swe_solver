\documentclass[paper=a4]{proc}
\usepackage[ngerman]{babel}

\pagenumbering{gobble}		%Disable page numbers

\title{Graphical Data Visualization Tool}
\author{George Bellamy, Christoph Honal, Felix Wieser}

\begin{document}
	\maketitle
	\thispagestyle{plain}	%Disable page numbers
	\section{Outline}
		This project is about implementing a graphical user interface to interactively visualize the output of the SWE-tsunami-simulation and provide basic analyzing functions for the tsunami waves. Additionally, the ability to export images, videos and graphs (i.e. cross-sections) are planned as well. The project is intended to be a standalone application which loads pre-calculated datasets.
	\section{Description}
		\subsection{GUI}
			The GUI consists of a 2D renderer and some control elements. It should support zoom/pan of the map and multiple layers of data to allow overlays.
 		\subsection{Data analysis}
			The following analysis-tools are planned at this time:
			\subsubsection*{Cell probe}
				Displays the raw cell data at a predefined point on the map, i.e. the pointers location or a fixed position.
			\subsubsection*{Arrival time calculator}
				Arrival time calculator: After the specification of a target location, the arrival time of the tsunami is calculated. Technically, the cells with a high water level are highlighted and a probe detects the first iteration, in which a highlighted cell is at or close to (coastal regions) this location.
			\subsubsection*{Coastal damage estimation}
				This feature is an extension of the arrival time calculator, except it probes all coastal regions and outputs the maximum wave height over time at each location.
			\subsection*{Cross-section analysis}
				In order to visualize a vertical profile, cross-sections can be very helpful. Two points must be selected on the map to specify the line along which the vertical profile of bathymetry and water height is then plotted.
		\subsection{Export}
			For increased usability the application should be capable to export images and maybe even videos of the scenario and graphs.
	\section{Implementation steps}
		The project is partitioned into several milestones, the essentials being the first three, and all others being additional features.
		\begin{itemize}
			\item Framework
			\begin{itemize}
				\item Using \emph{gtkmm} (with \emph{glade})
			\end{itemize}
			\item Height-map renderer
			\begin{itemize}
				\item Using \emph{openGL} pixelshaders
			\end{itemize}
			\item Data analysis functions
			\item Interactive GUI-Elements
			\item Add zoom/pan of the map
			\item Layer-based rendering
			\item Image and video export
			\begin{itemize}
				\item Using \emph{ffmpeg}
			\end{itemize}
		\end{itemize}
\end{document}